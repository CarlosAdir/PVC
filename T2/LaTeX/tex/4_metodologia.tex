\section{Metodologia}
\label{metodologia}
%\lipsum[1]

A metodologia consistiu em seguir os passos dos requisitos necessários. Para implementação primeiramente pensou-se em utilizar C++, mas por dificuldade na utilização de funções especificas da OpenCV optou-se por utilizar Python. Para escolha da câmera, optou-se por uma câmera de alta resolução com o foco autoajustável para verificar o efeito do foco sobre os parâmetros das matrizes $In$ e $Ex$ da Equação (\ref{eq:matrizes}).

Para o primeiro requisito, a metodologia consistiu em adaptar o código implementado em C++ no Projeto Demonstrativo 1.

Para o segundo e terceiro requisito, consistiu primeiramente em selecionar um tabuleiro de xadrez com $6\times 8$ interseções e cada quadrado com $30$ mm de lado, imprimiu-se e colou-se em um papelão espesso.

Com o tabuleiro, escolheu-se fazer 5 pacotes de medições com 25 imagens do tabuleiro em cada pacote. Cada imagem dentre as 25 eram bem similares entre si, enquanto imagem entre pacotes ficam em regiões separadas da imagem para haver uma abrangência maior. Para evitar perda de tempo lidando com o padrão frequentemente, planejou-se fazer um código para armazenar as imagens com padrão do tabuleiro assim que fossem capturadas.

Então, para verificar o efeito da distância sobre os parâmetros, fez-se o processo de 5 pacotes para 3 distâncias: a mais distante que ainda capturava o padrão, a menor distância que se reconhecia todos os pontos do tabuleiro, e uma distância intermediária. Apenas para a distância mais próxima era possível apenas 1 pacote de medição, visto que o tabuleiro ocupa a tela totalmente.

Para o requisito 4, foi mesclar os resultados obtidos nos requisitos 2 e 3 com o requisito 1. Como obteve-se os parâmetros intrínsecos, extrínsecos e a interface do requisito 1.

Para o código, seguiu-se a metodologia de utilizar funções já implementadas pela OpenCV para encontrar vertices do tabuleiro de xadrez, calcular os parâmetros das matrizes $In$ e $Ex$ da Equação (\ref{eq:matrizes}).