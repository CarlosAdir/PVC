\section{Metodologia}
\label{metodologia}
%\lipsum[1]

A metodologia consistiu em seguir os passos dos requisitos necessários. 

Para o primeiro requisito, a metodologia consistiu em adaptar o código implementado em C++ no Projeto Demonstrativo 1.

Para os requisitos posteriores, primeiramente precisou selecionar um tabuleiro de xadrez com $6\times 8$ interseções e cada quadrado com $30$ mm de lado, imprimiu-se e colou-se em um papelão espesso.

Com o tabuleiro, para o requisito 2, fez 5 pacotes de medições com 25 imagens do tabuleiro em cada pacote. Cada imagem em um pacote são bem similares entre si, enquanto imagem entre pacotes ficam em regiões separadas da imagem para haver uma abrangência maior. Para evitar perda de tempo capturando o padrão frequentemente, planejou-se fazer um código para armazenar as imagens com padrão do tabuleiro assim que fossem capturadas, o qual foi chamado de \textit{get\_images}.
Uma vez com $5 \times 25$ imagens do tabuleiro, utilizou-se a função \textit{findChessboardCorners}\cite{CameraCalibration} para encontrar os $48$ pontos para cada imagem. Com o conjunto de $25 \times 48$ pontos em cada pacote, utilizou-se a função \textit{calibrateCamera}\cite{CameraCalibration} para obter no total 5 matrizes e 5 coeficientes de distorção como parâmetros intrínsecos; %e os vetores de rotação e translação referentes aos parâmetros extrínsecos.
Com todos os parâmetros intrínsecos, pegou-se a média e desvio padrão das 5 matrizes intrínsecas(uma para cada pacote) e dos 5 coeficientes de distorção. 

Ja os parâmetros extrínsecos também foram obtidos pela função \textit{calibrateCamera}. Como tais parâmetros variam conforme a posição do tabuleiro, ou seja, entre os pacotes, não podemos pegar a média de varios pacotes como feito com os parâmetros intrínsecos. Para isso, pegou-se apenas um pacote de 25 imagens e calculou-se a posição média do padrão à câmera. Dos $25 \times 48$ pontos desse pacote, obteve-se $25$ vetores de rotação e $25$ vetores de translação bem próximos entre si, e depois pegou-se a média dos $25$ vetores.

Então, para verificar o efeito da distância do padrão sobre os parâmetros extrínsecos, fez-se o processo dito acima, variando apenas a distância do tabuleiro à câmera. As 3 distâncias escolhidas foram: a mais distante que ainda capturava o padrão, a menor distância que se reconhecia todos os pontos do tabuleiro, e uma distância intermediária. Apenas para a distância mais próxima era possível apenas 1 pacote de medição, visto que o tabuleiro ocupa a tela totalmente. Para cada distância, mediu-se com uma trena a distância entre a câmera.

Para o requisito 4, foi mesclar os resultados obtidos nos requisitos 2 e 3 com o requisito 1 pois a interface, os parâmetros intrínsecos e extrínsecos foram obtidos nos requisitos anteriores.

Para implementação do código, seguiu-se a estratégia de utilizar funções já implementadas pela OpenCV para encontrar vertices do tabuleiro de xadrez, calcular os parâmetros das matrizes $In$ e $Ex$ da Equação (\ref{eq:matrizes}). Para implementação primeiramente pensou-se em utilizar C++, mas por dificuldade na utilização de funções especificas da OpenCV optou-se por utilizar Python e utilizou-se um tutorial do OpenCV na utilização das funções\cite{CameraCalibrationTutorial}. Para escolha da câmera, optou-se por uma webcam, visto que outra câmera que era planejada de ser utilizada auto ajustava o seu foco e isso mudava os parâmetros intrínsecos frequentemente.