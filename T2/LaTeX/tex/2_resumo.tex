\begin{abstract}

A visão computacional precisa frequentemente relacionar o mundo virtual(imagens) com o mundo real(dimensões fisicas). Por isso, desenvolveu-se métodos para calibração de câmeras, que vinculam essas duas regiões. Esse trabalho tem por objetivo mostrar o processo de calibração de uma webcam comum de computador, partindo inicialmente da obtenção dos parâmetros intrínsecos, depois dos extrínsecos e como produto final realizar a medição de objetos através das imagens obtidas. Devido aos problemas de calibração encontrados, o trabalho não foi concluido a tempo embora tenha-se repetido o processo de calibração diversas vezes. Uma das grandes dificuldades encontradas na elaboração do trabalho consistiu na constante verificação do padrão e do código, pois não se sabia se os erros observados eram oriundos da captura de fotos ou do algoritmo que obtém os parâmetros a partir da imagem

\end{abstract}