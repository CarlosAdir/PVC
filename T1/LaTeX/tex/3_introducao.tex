\section{Introdução}
\label{sec:intro}

A disciplina Princípios de Visão Computacional da Universidade de Brasilia ensina diversas coisas relacionadas à visão computacional, que em resumo consiste em transformar imagens em dados. Para isso, a disciplina utiliza da biblioteca OpenCV\cite{OpenCVSite} para implementação de algoritmos tanto para auxiliar na aquisição , quanto no tratamento de dados.

Para aplicações futuras na disciplina, é necessário que o aluno se sinta confortavel na utilização das funcionabilidades da biblioteca OpenCV e esse projeto demonstrativo tem por objetivo aproximar o contato do aluno à biblioteca OpenCV.

As especificações do projeto funcionam de forma gradual, ou seja, os requisitos posteriores utilizam boa parte dos requisitos anteriores. As especificações consistem, em resumo de:

\begin{enumerate}
\item Fazer o upload de uma imagem armazenada e fazer uma interface interativa com cliques do mouse.
\item Com o procedimento anterior, alterar determinados pixeis e deixa-los vermelhos
\item Semelhante ao requisito anterior, mas com video no lugar de uma imagem.
\item O mesmo que o requisito anterior, mas utilizando uma câmera no lugar do video. 
\end{enumerate}

Para implementação dos 4 algoritmos que cumpram os requisitos, utilizou-se C++ como linguagem de implementação. Para aprendizado das funções e biblioteca foi utilizado tanto a páǵina oficial da OpenCV\cite{OpenCVSite}, quanto um tutorial disponibilizado na internet\cite{OpenCV-SFR}.
